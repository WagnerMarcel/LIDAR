%%%%%%%%%%%%%%%%%%%%%%%%%%%%%%%%%%%%%%%%%%%%%%%%%%%%%%%%%%%%%%%%%%%%%%%%%%%%%%%
%                                   Einstellungen
%
% Hier können alle relevanten Einstellungen für diese Arbeit gesetzt werden.
% Dazu gehören Angaben u.a. über den Autor sowie Formatierungen.
%
%
%%%%%%%%%%%%%%%%%%%%%%%%%%%%%%%%%%%%%%%%%%%%%%%%%%%%%%%%%%%%%%%%%%%%%%%%%%%%%%%


%%%%%%%%%%%%%%%%%%%%%%%%%%%%%%%%%%%% Sprache %%%%%%%%%%%%%%%%%%%%%%%%%%%%%%%%%%%
%% Aktuell sind Deutsch und Englisch unterstützt.
%% Es werden nicht nur alle vom Dokument erzeugten Texte in
%% der entsprechenden Sprache angezeigt, sondern auch weitere
%% Aspekte angepasst, wie z.B. die Anführungszeichen und
%% Datumsformate.
\setzesprache{de} % oder en
%%%%%%%%%%%%%%%%%%%%%%%%%%%%%%%%%%%%%%%%%%%%%%%%%%%%%%%%%%%%%%%%%%%%%%%%%%%%%%%%

%%%%%%%%%%%%%%%%%%%%%%%%%%%%%%%%%%% Angaben  %%%%%%%%%%%%%%%%%%%%%%%%%%%%%%%%%%%
%% Die meisten der folgenden Daten werden auf dem
%% Deckblatt angezeigt, einige auch im weiteren Verlauf
%% des Dokuments.
\setzemartrikelnr{-todo-}
\setzekurs{TEL16GR2}
\setzetitel{Entwurf, entwicklung und validierung eines \ac{LIDAR} Systems}
\setzedatumAbgabe{-todo-}
\setzefirma{Robert Bosch GmbH}
\setzefirmenort{Stuttgart}
\setzeabgabeort{Stuttgart}
%\setzeabschluss{Bachelor of Engineering}
\setzestudiengang{Elektrotechnik, Elektronik}
\setzedhbw{Stuttgart}
\setzebetreuer{Klaus Gosger}
\setzezeitraum{-todo-}
\setzearbeit{Seminararbeit}
\setzeautor{Alexander Kehrer \& Marcel Wagner}

\inhalttrue                 % auskommentieren oder ändern zu \inhaltfalse, falls kein Inhaltsverzeichnis eingefügt werden soll
%\unterschriftenblatttrue    % auskommentieren oder ändern zu \unterschriftenblattfalse, falls kein Unterschriftenblatt eingefügt werden soll
\selbsterkltrue             % auskommentieren oder ändern zu \selbsterklfalse, wenn keine Selbstständigkeitserklärung benötigt wird
%\sperrvermerktrue           % auskommentieren oder ändern zu \sperrvermerkfalse, wenn kein Sperrvermerk benötigt wird
\abkverztrue                % auskommentieren oder ändern zu \abkverzfalse, wenn kein Abkürzungsverzeichnis benötigt wird
\abbverztrue                % auskommentieren oder ändern zu \abbverzfalse, wenn kein Abbildungsverzeichnis benötigt wird
\tableverztrue              % auskommentieren oder ändern zu \tableverzfalse, wenn kein Tabellenverzeichnis benötigt wird
\listverztrue               % auskommentieren oder ändern zu \listverzfalse, wenn kein Listingsverzeichnis benötigt wird
\formelverztrue             % auskommentieren oder ändern zu \formelverzfalse, wenn kein Formelverzeichnis benötigt wird
\abstracttrue               % auskommentieren oder ändern zu \abstractfalse, wenn kein Abstract gewünscht ist
\bothabstractstrue          % auskommentieren oder ändern zu \bothabstractsfalse, wenn nur der Abstract in der Hauptsprache eingefügt werden soll
\appendixtrue               % auskommentieren oder ändern zu \appendixfalse, wenn kein Anhang gewünscht ist
\literaturtrue              % auskommentieren oder ändern zu \literaturfalse, wenn kein Literaturverzeichnis gewünscht ist (\appendixtrue muss gesetzt sein!)
\glossartrue                % auskommentieren oder ändern zu \glossarfalse, wenn kein Glossar gewünscht ist (\appendixtrue muss gesetzt sein!)
\watermarkfalse             % auskommentieren oder ändern zu \watermarktrue, wenn Wasserzeichen auf dem Titelblatt eingefügt werden soll

\refWithPagesfalse          % ändern zu \refWithPagestrue, wenn die Seitenzahl bei Verweisen auf Kapitel engefügt werden sollen


% Angabe des roten/gelben Punktes auf dem Titelblatt zur Kennzeichnung der Vertraulichkeitsstufe.
% Mögliche Angaben sind \yellowdottrue und \reddottrue. Werden beide angegeben, wird der rote Punkt gezeichnet.
% Wird keines der Kommandos angegeben, wird kein Punkt gezeichnet
%\yellowdottrue

%%%%%%%%%%%%%%%%%%%%%%%%%%%%%%%%%%%%%%%%%%%%%%%%%%%%%%%%%%%%%%%%%%%%%%%%%%%%%%%%

%%%%%%%%%%%%%%%%%%%%%%%%%%%% Literaturverzeichnis %%%%%%%%%%%%%%%%%%%%%%%%%%%%%%
%% Bei Fehlern während der Verarbeitung bitte in ads/header.tex bei der
%% Einbindung des Pakets biblatex (ungefähr ab Zeile 110,
%% einmal für jede Sprache), biber in bibtex ändern.
\newcommand{\ladeliteratur}{%
\addbibresource{bibliographie.bib}
%\addbibresource{weitereDatei.bib}
}

%% Zitierstil
%% siehe: http://ctan.mirrorcatalogs.com/macros/latex/contrib/biblatex/doc/biblatex.pdf (3.3.1 Citation Styles)
%% mögliche Werte z.B numeric-comp, alphabetic, authoryear
\setzezitierstil{numeric-comp}
%%%%%%%%%%%%%%%%%%%%%%%%%%%%%%%%%%%%%%%%%%%%%%%%%%%%%%%%%%%%%%%%%%%%%%%%%%%%%%%%

%%%%%%%%%%%%%%%%%%%%%%%%%%%%%%%%% Layout %%%%%%%%%%%%%%%%%%%%%%%%%%%%%%%%%%%%%%%
%% Verschiedene Schriftarten
% laut nag Warnung: palatino obsolete, use mathpazo, helvet (option scaled=.95), courier instead
\setzeschriftart{lmodern} % palatino oder goudysans, lmodern, libertine

%% Abstand vor Kapitelüberschriften zum oberen Seitenrand
\setzekapitelabstand{20pt}

%% Spaltenabstand
\setzespaltenabstand{10pt}
%%Zeilenabstand innerhalb einer Tabelle
\setzezeilenabstand{1.5}
%%%%%%%%%%%%%%%%%%%%%%%%%%%%%%%%%%%%%%%%%%%%%%%%%%%%%%%%%%%%%%%%%%%%%%%%%%%%%%%%