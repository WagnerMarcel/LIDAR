%!TEX root = ../dokumentation.tex
\chapter{Stand der Technik}\label{chap:stand_der_technik}
Im folgenden Kapitel sollen bereits vorhandene \ac{LIDAR} Systeme betrachtet werden.\\
Dabei wird ein Vergleich verschiedener Systeme, sowie die Analyse deren Funktionen und die mögliche Übertragung dieser Funktionen auf das in dieser Arbeit entstehende System erfolgen.\\
Die Systeme, welche verglichen werden, sind meist für die Anwendung im Bereich der Geodäsie entwickelt und ermöglichen ein schnelles, präzises und dreidimensionales vermessen von Landschaften oder Baustellen. Zusätzlich werden auch Systeme, welche für Logistik oder Raumvermessung entwickelt worden sind, verglichen. \\

\section{Vergleich der Sensoren}
Bei Betrachtung der Tabelle \ref{vergleich_sensoren} kann beobachtet werden, dass je nach Anwendungsgebiet des Sensors unterschiedliche Parameter ausschlaggebend für den Entwurf des Systems sind. Zudem kann betrachtet werden, dass Systeme in dieser Kategorie sehr preisintensiv sind. \\
Vor allem sticht die maximale Messrate der verschiedenen Sensoren besonders hervor. Diese ist dabei stark vom Einsatzgebiet abhängig. Gerade bei Systemen, welche für den Außenraum konzipiert sind, ist eine hohe Messrate gefordert, damit in einer annehmbaren Zeit ein möglichst genaues Abbild der Umgebung erstellt werden kann. Auch ist eine große Reichweite von Vorteil, damit auch Objekte in großer Entfernung noch aufgezeichnet werden können. \\
Zudem kann beobachtet werden, dass die meisten der Systeme das \ac{ToF} Prinzip verwenden. Dadurch werden zum einen hohe Messfrequenzen möglich, zum anderen lassen sich auch Zusatzfunktionen wie z.B. die Multi-Echo Auswertung implementieren. \\
\newpage
Die großen Unterschiede der Auflösung sind ebenfalls auf die Anforderungen aus den verschiedenen Anwendungsgebieten zurückzuführen. Beim Vermessen von Gebäuden oder Landschaften ist eine hohe Auflösung gerade für die Planung sehr wichtig. Bei der Logistik kommt es allerdings eher auf einen geringen Preis an, als auf eine millimetergenaue Auflösung, da ohnehin immer ein Abstand zum Hindernis gewahrt werden muss.\\
Die große Preisspanne der verschiedenen Systeme ist den verschiedenen Anwendungsgebieten geschuldet, da die Sensoren dort unterschiedlich oft eingesetzt werden. So müssen für die Automatisierung einer Logistikzentrale beispielsweise viele der Systeme verbaut werden, während für das Erfassen der Außenumgebung ein einzelnes System vollkommen ausreichend ist.

\begin{landscape}
	\begin{table}[H]
		\centering
		\caption{Vergleich verschiedener \ac{3D} \ac{LIDAR} Systeme}\label{vergleich_sensoren}
		\begin{tabular}{|c|c|c|c|c|}
			\hline
			& \textbf{Sick MRS6000 \cite{sick}} & \textbf{\begin{tabular}{@{}c@{}}Ocular\\Robotics RE05\end{tabular} \cite{ocular}} & \textbf{\begin{tabular}{@{}c@{}}Leica Scan\\Station P50\end{tabular} \cite{leica}} & \textbf{Artec Ray \cite{artec}}\\
			\hline
			\textbf{Anwendungsgebiet} & Logistik & Innenraum & Außenraum & Außenraum \\
			\hline
			\textbf{Reichweite [m]} &  200 & 160 & >1000 & 110   \\
			\hline
			\textbf{Messrate [Hz]} &  10 & 30000 & 1000000 & 208000   \\
			\hline
			\textbf{Auflösung [mm]} & 125 & 50 & 3 & 0.9   \\
			\hline
			\textbf{Verfahren} &  
				\begin{tabular}{@{}c@{}} 
					\ac{ToF} \& \\ Bewegende Polygonspiegel 
				\end{tabular} &  
				- &  
				\begin{tabular}{@{}c@{}} 
					\ac{ToF} \& \\
					Rotierender \\ 
					Sender/Empfänger
				\end{tabular} & 
				\begin{tabular}{@{}c@{}} 
					Phasenverschiebung \& \\ 
					rotierender \\
					Sender/Empfänger
				\end{tabular} 
				 \\
			\hline
			\textbf{Zusatzfunktionen} & 
				\begin{tabular}{@{}c@{}} 
					Multi-Echo \\
					Auswertung
				\end{tabular} &
				\begin{tabular}{@{}c@{}} 
					Horizontales Einstellen \\ des Messbereichs
				\end{tabular} &  
				\begin{tabular}{@{}c@{}} 
					Farbkamera \\
					
				\end{tabular} &  
				
			\\
			\hline
			\textbf{Preis [Euro]} & 7000 & - & 125500 & 50000
			\\\hline
		\end{tabular}
	\end{table}
\end{landscape}

\section{Erkenntnisse}
Die größte Erkenntnis aus den Ergebnissen des Vergleichs ist, dass man immer Kompromisse bei der Auslegung eines Systems eingehen muss. Vor allem ist der Kostenpunkt der größte limitierende Faktor, da dieser vorgibt, welche Methode verwendet wird und welche Messraten möglich sind. \\
Zudem kann man aus den Recherchen die Erkenntnis erlangen, dass es nur wenige Systeme für die Vermessung von Innenräumen gibt. Zudem werden Innenraumvermessungen oft nur als Dienstleistung angeboten. Außerdem sind bestehende Systeme sehr kostenintensiv und daher nur für Wenige eine Option.\\
Da im Rahmen dieser Arbeit ein kostengünstiges System entwickelt werden soll, müssen dafür verschiedene Kompromisse eingegangen werden, welche in den folgenden Kapiteln erläutert werden.
