%!TEX root = ../dokumentation.tex

% Marcel
\chapter{Grundlagen Elektronik}\label{chap:grundlagen_et}
\section{Photodioden}\label{sec:photodioden}
Um Licht zu detektieren werden meist Photodioden verwendet. Diese arbeiten nach einem relativ einfachen Prinzip.
Eine p-n-Diode wird in Sperrrichtung betrieben, durch die Angelegte Spannung entsteht eine Sperrschicht. Wenn nun Photonen auf die offene, starke p-Dotierung treffen werden dort durch den Photoeffekt Ladungsträger erzeugt (Abbildung: \ref{photodiode}). Wenn diese nun durch Diffusion bis zur Sperrschicht gelangen, driften die Ladungsträger entgegen der Sperrspannung in die jeweiligen Raumladungszonen, dies ist als Strom messbar. \cite{Photodiode_spektrum}
\begin{figure}[H]
	\centering
	\includegraphics[width=0.75\textwidth]{images/GrundlagenLaserentfernungsmessung/Photodiode}
	\caption{Schematischer Aufbau einer Photodiode \cite{Photodiode_spektrum} ($p^+$ starke p-Dotierung)}
	\label{photodiode}
\end{figure}
Dieser Effekt tritt allerdings nur auf, wenn die Photonen eine Energie großer als die des Bandabstandes des verwendeten Halbleiters aufweisen. Hierbei ist zudem pro eintreffendem Photon nur ein sehr geringer Stromimpuls messbar, daher ist diese Art von Diode für \ac{LIDAR} Anwendungen nicht brauchbar.
\subsection{\acf{APD}}\label{subsec:apd}
Um einzelne Photonen detektieren zu können wird eine spezielle Form der Photodiode verwendet. Die sogenannte \ac{APD}. Die \ac{APD} hat im Gegensatz zur herkömmlichen Photodiode zweit weitere Schichten. Hinzu zur n-Dotierten und stark p-Dotierten Schicht kommen nun eine schwach p-Dotierte (oder intrinsische) und eine "normal" p-Dotierte Schicht (Abbildung: \ref{apd}). 
\begin{figure}[H]
	\centering
	\includegraphics[width=0.75\textwidth]{images/GrundlagenLaserentfernungsmessung/APD}
	\caption{Schematischer Aufbau einer \ac{APD} \cite{APD_Scematic} ($p^+$ starke p-Dotierung, $p^- (\pi)$ schwache (intrinsische) p-Dotierung, $n^+$ starke n-Dotierung) (1 - Metallkontakte, 2 - Entspiegelung)}
	\label{apd}
\end{figure}
Wenn Photonen nun in die $\pi$ Zone gelangen, werden dort Landungsträger erzeugt, diese werden gleich wie bei der regulären Photodiode getrennt, Löcher wandern Richtung $p^+$-Zone und Elektronen Richtung $n^+$-Zone. Durch die stärker Dotierte p-Zone, und somit höhere Feldstärke, werden die Elektronen beschleunigt und es entsteht eine Stoßionisation. \ac{APD} werden mit sehr hohen Sperrspannungen $\sim$100V, nahe der Durchbruchspannung betrieben. \cite{SPAD_mamamatsu} \\
Wenn die \ac{APD} oberhalb der Durchbruchsspannung betrieben wird, setzt sich die Stoßionisation lawinenartig fort (Avalanche-Effekt) und es entstehen Verstärkungsfaktoren von einigen Millionen. \ac{APD} welche speziell für den Betrieb oberhalb der Durchbruchspannung ausgelegt sind werden auch \ac{SPAD} genannt. Mittels diesem Effekt kann man einzelne Photonen nachweisen, da jedes Photon einen kurzen detektierbaren Stromimpuls erzeugt. Bei der anordnung vieler solcher \acp{SPAD} in einem Array können viele einzelne Photonen präzise nachgewiesen werden. \cite{SPAD_elmer}\\
\subsection{Lateral auflösende Photodiode}
Die lateral auflösende Photodiode, auch \ac{PSD} genannt, verwendet mehrere aneinander geschlossene Photodioden zur Bestimmung der Position eines Lichtpunktes.
\begin{figure}[H]
	\centering
	\includegraphics[width=0.75\textwidth]{images/GrundlagenLaserentfernungsmessung/PSD}
	\caption{Aufbau einer \ac{PSD} \cite{APD_Scematic}}
\end{figure}
Wenn man nun jeweils den Photostrom der Dioden miteinander Vergleicht kann man eine sehr präzise Auskunft über die Position des Lichtpunkts geben.
\begin{equation}\formelentry{Position des Lichtpunkts einer \ac{PSD}}
	x = \frac{I_1 - I_2}{I_1 + I_2}
\end{equation}
\begin{flalign*}
	&I_1 = \text{Strom aus der linken Photodiode}\left[A \right]&\\
	&I_2 = \text{Strom aus der rechten Photodiode} \left[A \right]&\\
	&x = \text{Relative Position des Lichtpunktes}&
\end{flalign*}
Die Differenz der Ströme wird hier auf den Gesamtstrom Normiert, dies hat zur Folge, dass die Position unabhängig von der Intensität des Lichts wird.\\
Um diese Auswertung zu realisieren ist lediglich eine Transimpedanzverstärkerschaltung gekoppelt mit einer Komperatorschaltung nötig, mit welcher die Spannungen verglichen werden können.\\
Ein sehr großer Vorteil der \ac{PSD} gegenüber anderer Methoden zur Feststellung der Position eines Lichtpunkts ist, dass die \ac{PSD} innerhalb von Nanosekunden reagieren und das mit einer Sehr genauen Auflösung.\cite{psd}


% Alexander
\section{Schrittmotoren}\label{sec:schrittmotoren}

Beim Schrittmotor handelt es sich um einen Synchronmotor. Innerhalb des feststehenden Stators befindet sich ein drehend gelagerter Rotor. Wird ein Schrittmotor entsprechend angesteuert, dreht sich der Rotor um einen bestimmten Drehwinkel weiter. Durch mehrere Schritte kann der Rotor um jeden Drehwinkel, der einem Vielfachen des minimalen Drehwinkels entspricht, gedreht werden.\cite{Schrittmotor} \\
Man unterscheidet drei Bauformen von Schrittmotoren: \\
Der Reluktanz-Schrittmotor ist die älteste Bauweise von Schrittmotoren. Dabei besteht der gezahnte Rotor sowie der Strator aus weichmagnetischen Material. Durch Anlegen von Strömen bilden sich magnetische Felder aus und der Rotor dreht sich. Durch die fehlenden Permanentmagneten hat der Motor jedoch kein Rastermoment im ausgeschalteten Zustand.\\
Beim Permanentmagnet-Schrittmotor besteht der Strator aus Weicheisen und der Rotor aus Permanentmagneten. Durch geschickte Bestromung des Strators, wird der Rotor immer so ausgerichtet, dass eine Drehbewegung entsteht. In dieser Bauform ist die Auflösung durch die limitierte Anzahl von Polen begrenzt.\cite{Schrittmotor_Bauformen}\\
Beim Hybridschrittmotor werden die Vorzüge der beiden genannten Motorarten vereint. Der Stator besteht aus gezahntem Weichmetall. Der Rotor besteht aus einem Permanentmagneten mit axialer Magnetfeldausrichtung. Darauf werden zwei fein gezackte, weichmagnetische Dynamobleche angebracht. Diese sind zueinander verdreht, sodass es zu einer Polteilung kommt und sich Süd- und Nordpole abwechseln. Der Hybridmotor zeichnet sich durch gutes Drehmoment und eine gute Auflösung aus.\cite{Schrittmotor_Bauformen}

Schrittmotoren werden aufgrund ihrer Genauigkeit meist für die Positionierung verwendet. Auch ohne Sensoren zur Positionsrückgabe können diese genutzt werden. Die Ansteuerung erfolgt über eine aktive Phasensteuerung der einzelnen Motorphasen. 

\todo{Bild einfügen?}