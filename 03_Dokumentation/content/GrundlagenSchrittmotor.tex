%!TEX root = ../dokumentation.tex


\chapter{Grundlagen}

\section{Schrittmotoren}

Beim Schrittmotor handelt es sich um einen Synchronmotor. Innerhalb des feststehenden Strators befindet sich ein drehend gelagerter Rotor. Wird ein Schrittmotor entsprechend angesteuert, dreht sich der Rotor um einen bestimmten Drehwinkel weiter. Durch mehrere Schritte kann der Rotor um jeden Drehwinkel, der einem Vielfachen des minimalen Drehwinkels entspricht, gedreht werden. \\
Man unterscheidet drei Bauformen von Schrittmotoren: \\
Der Reluktanz-Schrittmotor ist die älteste Bauweise von Schrittmotoren. Dabei besteht der gezahnte Rotor sowie der Strator aus weichmagnetischen Material. Durch Anlegen von Strömen bilden sich magnetische Felder aus und der Rotor dreht sich. Durch die fehlenden Permanentmagneten hat der Motor jedoch kein Rastermoment im ausgeschalteten Zustand.\\
Beim Permanentmagnet-Schrittmotor besteht der Strator aus Weicheisen und der Rotor aus Permanentmagneten. Durch geschickte Bestromung des Strators, wird der Rotor immer so ausgerichtet, dass eine Drehbewegung entsteht. In dieser Bauform ist die Auflösung durch die limitierte Anzahl von Polen begrenzt.\\

Beim Hybridschrittmotor werden die Vorzüge der beiden genannten Motorarten vereint. Der Strator besteht aus gezahntem Weichmetall. Der Rotor besteht aus einem Permanentmagneten mit axialer Magnetfeldausrichtung. Darauf werden zwei fein gezackte, weichmagnetische Dynamobleche angebracht. Diese sind zueinander verdreht, sodass es zu einer Polteilung kommt und sich Süd- und Nordpole abwechseln. Der Hybridmotor zeichnet sich durch gutes Drehmoment und eine gute Auflösung aus.

--ToDO: Quellen und Bilder einfügen, allgemeine Funktion beschreiben --