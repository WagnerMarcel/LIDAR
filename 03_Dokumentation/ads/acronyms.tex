%!TEX root = ../dokumentation.tex
%nur verwendete Akronyme werden letztlich im Abkürzungsverzeichnis des Dokuments angezeigt
%Verwendung: 
%		\ac{Abk.}   --> fügt die Abkürzung ein, beim ersten Aufruf wird zusätzlich automatisch die ausgeschriebene Version davor eingefügt bzw. in einer Fußnote (hierfür muss in header.tex \usepackage[printonlyused,footnote]{acronym} stehen) dargestellt
%		\acs{Abk.}   -->  fügt die Abkürzung ein
%		\acf{Abk.}   --> fügt die Abkürzung UND die Erklärung ein
%		\acl{Abk.}   --> fügt nur die Erklärung ein
%		\acp{Abk.}  --> gibt Plural aus (angefügtes 's'); das zusätzliche 'p' funktioniert auch bei obigen Befehlen
%	siehe auch: http://golatex.de/wiki/%5Cacronym
%\acro{BSP}{Board Support Package} % Beispielabkürzung

\acro{3D}	{Dreidimensional}
\acro{APD}	{Avalanche Photo Diode}
\acro{CAD}	{Computer Aided Design}
\acro{CNC}	{Computerized Numerical Control}
\acro{csv}	{comma separated values}
\acro{GPIO}	{General Purpose Input Output}
\acro{GUI}	{Graphical User Interface}
\acro{IC}	{Integrated Circuit}
\acro{I$^{2}$C}	{Inter-Integrated Circuit}
\acro{IP}	{Internet Protocol}
\acro{LAN}	{Local Area Network}
\acro{LED}	{Light Emmiting Diode}
\acro{LIDAR}{Light Distance and Ranging}
\acro{NEMA}	{National Electrical Manufacturers Association}
\acro{PSD}	{Position Sensitive Detector}
\acro{SPAD}	{Single Photon Avalanche Diode}
\acro{SPI}	{Serial Peripheral Interface}
\acro{ToF}	{Time of Flight}
\acro{UART}	{Universal Asynchronous Receiver Transmitter}
\acro{WLAN}	{Wireless Local Area Network}