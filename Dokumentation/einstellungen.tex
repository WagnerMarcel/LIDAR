%%%%%%%%%%%%%%%%%%%%%%%%%%%%%%%%%%%%%%%%%%%%%%%%%%%%%%%%%%%%%%%%%%%%%%%%%%%%%%%
%                                   Einstellungen
%
% Hier k�nnen alle relevanten Einstellungen f�r diese Arbeit gesetzt werden.
% Dazu geh�ren Angaben u.a. �ber den Autor sowie Formatierungen.
%
%
%%%%%%%%%%%%%%%%%%%%%%%%%%%%%%%%%%%%%%%%%%%%%%%%%%%%%%%%%%%%%%%%%%%%%%%%%%%%%%%


%%%%%%%%%%%%%%%%%%%%%%%%%%%%%%%%%%%% Sprache %%%%%%%%%%%%%%%%%%%%%%%%%%%%%%%%%%%
%% Aktuell sind Deutsch und Englisch unterst�tzt.
%% Es werden nicht nur alle vom Dokument erzeugten Texte in
%% der entsprechenden Sprache angezeigt, sondern auch weitere
%% Aspekte angepasst, wie z.B. die Anf�hrungszeichen und
%% Datumsformate.
\setzesprache{de} % oder en
%%%%%%%%%%%%%%%%%%%%%%%%%%%%%%%%%%%%%%%%%%%%%%%%%%%%%%%%%%%%%%%%%%%%%%%%%%%%%%%%

%%%%%%%%%%%%%%%%%%%%%%%%%%%%%%%%%%% Angaben  %%%%%%%%%%%%%%%%%%%%%%%%%%%%%%%%%%%
%% Die meisten der folgenden Daten werden auf dem
%% Deckblatt angezeigt, einige auch im weiteren Verlauf
%% des Dokuments.
\setzemartrikelnr{-todo-}
\setzekurs{-todo-}
\setzetitel{-todo-}
\setzedatumAbgabe{-todo-}
\setzefirma{Robert Bosch GmbH}
\setzefirmenort{-todo-}
\setzeabgabeort{Stuttgart}
%\setzeabschluss{Bachelor of Engineering}
\setzestudiengang{-todo-}
\setzedhbw{Stuttgart}
\setzebetreuer{-todo-}
\setzezeitraum{-todo-}
\setzearbeit{Praktikumsarbeit}
\setzeautor{-todo-}

\inhalttrue                 % auskommentieren oder �ndern zu \inhaltfalse, falls kein Inhaltsverzeichnis eingef�gt werden soll
\unterschriftenblatttrue    % auskommentieren oder �ndern zu \unterschriftenblattfalse, falls kein Unterschriftenblatt eingef�gt werden soll
\selbsterkltrue             % auskommentieren oder �ndern zu \selbsterklfalse, wenn keine Selbstst�ndigkeitserkl�rung ben�tigt wird
\sperrvermerktrue           % auskommentieren oder �ndern zu \sperrvermerkfalse, wenn kein Sperrvermerk ben�tigt wird
\abkverztrue                % auskommentieren oder �ndern zu \abkverzfalse, wenn kein Abk�rzungsverzeichnis ben�tigt wird
\abbverztrue                % auskommentieren oder �ndern zu \abbverzfalse, wenn kein Abbildungsverzeichnis ben�tigt wird
\tableverztrue              % auskommentieren oder �ndern zu \tableverzfalse, wenn kein Tabellenverzeichnis ben�tigt wird
\listverztrue               % auskommentieren oder �ndern zu \listverzfalse, wenn kein Listingsverzeichnis ben�tigt wird
\formelverztrue             % auskommentieren oder �ndern zu \formelverzfalse, wenn kein Formelverzeichnis ben�tigt wird
\abstracttrue               % auskommentieren oder �ndern zu \abstractfalse, wenn kein Abstract gew�nscht ist
\bothabstractstrue          % auskommentieren oder �ndern zu \bothabstractsfalse, wenn nur der Abstract in der Hauptsprache eingef�gt werden soll
\appendixtrue               % auskommentieren oder �ndern zu \appendixfalse, wenn kein Anhang gew�nscht ist
\literaturtrue              % auskommentieren oder �ndern zu \literaturfalse, wenn kein Literaturverzeichnis gew�nscht ist (\appendixtrue muss gesetzt sein!)
\glossartrue                % auskommentieren oder �ndern zu \glossarfalse, wenn kein Glossar gew�nscht ist (\appendixtrue muss gesetzt sein!)
\watermarkfalse             % auskommentieren oder �ndern zu \watermarktrue, wenn Wasserzeichen auf dem Titelblatt eingef�gt werden soll

\refWithPagesfalse          % �ndern zu \refWithPagestrue, wenn die Seitenzahl bei Verweisen auf Kapitel engef�gt werden sollen


% Angabe des roten/gelben Punktes auf dem Titelblatt zur Kennzeichnung der Vertraulichkeitsstufe.
% M�gliche Angaben sind \yellowdottrue und \reddottrue. Werden beide angegeben, wird der rote Punkt gezeichnet.
% Wird keines der Kommandos angegeben, wird kein Punkt gezeichnet
\yellowdottrue

%%%%%%%%%%%%%%%%%%%%%%%%%%%%%%%%%%%%%%%%%%%%%%%%%%%%%%%%%%%%%%%%%%%%%%%%%%%%%%%%

%%%%%%%%%%%%%%%%%%%%%%%%%%%% Literaturverzeichnis %%%%%%%%%%%%%%%%%%%%%%%%%%%%%%
%% Bei Fehlern w�hrend der Verarbeitung bitte in ads/header.tex bei der
%% Einbindung des Pakets biblatex (ungef�hr ab Zeile 110,
%% einmal f�r jede Sprache), biber in bibtex �ndern.
\newcommand{\ladeliteratur}{%
\addbibresource{bibliographie.bib}
%\addbibresource{weitereDatei.bib}
}

%% Zitierstil
%% siehe: http://ctan.mirrorcatalogs.com/macros/latex/contrib/biblatex/doc/biblatex.pdf (3.3.1 Citation Styles)
%% m�gliche Werte z.B numeric-comp, alphabetic, authoryear
\setzezitierstil{alphabetic}
%%%%%%%%%%%%%%%%%%%%%%%%%%%%%%%%%%%%%%%%%%%%%%%%%%%%%%%%%%%%%%%%%%%%%%%%%%%%%%%%

%%%%%%%%%%%%%%%%%%%%%%%%%%%%%%%%% Layout %%%%%%%%%%%%%%%%%%%%%%%%%%%%%%%%%%%%%%%
%% Verschiedene Schriftarten
% laut nag Warnung: palatino obsolete, use mathpazo, helvet (option scaled=.95), courier instead
\setzeschriftart{lmodern} % palatino oder goudysans, lmodern, libertine

%% Abstand vor Kapitel�berschriften zum oberen Seitenrand
\setzekapitelabstand{20pt}

%% Spaltenabstand
\setzespaltenabstand{10pt}
%%Zeilenabstand innerhalb einer Tabelle
\setzezeilenabstand{1.5}
%%%%%%%%%%%%%%%%%%%%%%%%%%%%%%%%%%%%%%%%%%%%%%%%%%%%%%%%%%%%%%%%%%%%%%%%%%%%%%%%